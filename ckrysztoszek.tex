\documentclass[a4paper, 11pt]{article}
\usepackage[protrusion=true,expansion=true]{microtype} 
\usepackage{graphicx} 
\usepackage{wrapfig} 
\usepackage{lipsum}
\usepackage [MeX]{polski}
\usepackage [utf8]{inputenc}
\usepackage{mathpazo} 
\usepackage{sidecap}
\usepackage[T1]{fontenc} 
\linespread{1.05} %
\makeatletter
\renewcommand{\maketitle}{ 
\begin{flushright} 
{\LARGE\@title} 
\vspace{40pt}
{\large\@author} 
\\\@date 
\vspace{50pt} 
\end{flushright}
}

\title{\textbf{Issac Newton}
Bigorafia} 
\author{\textsc{Cezary Krysztoszek} 
\\{\textit{Uniwersytet Gdański}}} 
\begin{document}
\maketitle 

\section*{Informacje Ogólne}
\begin{wrapfigure}{r}{0.5\textwidth}
\begin{center}
\vspace{-20pt}
\includegraphics[width=0.48\textwidth]I
\end{center}
\caption{Issac Newton}
\end{wrapfigure}
Isaac Newton (ur. 4 stycznia 1643 w Woolsthorpe-by-Colsterworth, 31 marca 1727 w Kensington) – angielski fizyk, matematyk, astronom, filozof, historyk, badacz Biblii i alchemik. Odkrywca zasad dynamiki.
W swoim słynnym dziele Philosophiae naturalis principia mathematica (1687 r.) przedstawił prawo powszechnego ciążenia oraz prawa ruchu, leżące u podstaw mechaniki klasycznej. Niezależnie od Gottfrieda Leibniza przyczynił się do rozwoju rachunku różniczkowego i całkowego. Opis zjawisk fizycznych za pomocą równań różniczkowych jest do dzisiaj cechą fizyki.

\section*{Nauka}
\subparagraph{Prawa} 
\begin{enumerate}
\item Prawo powszechnego ciążenia 
\newline
\begin{math}
F^i = G\frac{m^{}_{1}{m^{}_{2}}}{r^2}e^i
\end{math}
 \item Prawa ruchu
 \newline
 \begin{math}
 F=\frac{dp}{dt}
\end{math}
\end{enumerate}
\section*{Odkrycia Naukowe}
\subparagraph{Fizyka\newline}
Od 1670 do 1672 wykładał optykę. W tym czasie badał załamanie (refrakcję) światła, pokazując, że pryzmat może rozszczepić białe światło w widmo barw, a potem soczewka i drugi pryzmat powodują uzyskanie białego światła ponownie z kolorowego widma. Na tej podstawie wywnioskował, że każdy refraktor (teleskop soczewkowy) będzie posiadał wadę polegającą na rozszczepieniu światła (aberracja chromatyczna), aby uniknąć tego problemu zaprojektował własny typ teleskopu wykorzystujący zwierciadło zamiast soczewki znany później jako teleskop Newtona (teleskop zwierciadlany). Później, kiedy dostępne stały się szkła o różnych własnościach dyspersyjnych, problem ten rozwiązano, stosując soczewki achromatyczne. W 1671 Royal Society poprosiło o demonstrację jego teleskopu zwierciadlanego. Zainteresowanie to zachęciło Newtona do opublikowania notatek pt. On Colour, które później rozwinął w większe dzieło pt. Opticks. Kiedy Robert Hooke skrytykował niektóre z pomysłów Newtona, ten obraził się do tego stopnia, że wycofał się z publicznej debaty.

\subparagraph{Optyka\newline}
W 1679 Newton powrócił do swojej pracy nad grawitacją i jej wpływem na orbity planet, odwołując się do praw Keplera. Swoje wyniki opublikował w De motu corporum (1684). Obejmowała ona początki praw ruchu, które zostały szerzej omówione w Principiach
\section*{Opublikowane Ksiązki}
\begin{tabular}{|r|c|c|}
\hline 
N. & Tytuł & Rok wydania \\
\hline
1   & Method of Fluxions & 1671 \\
\hline
2   & De motu corporum in gyrum & 1684 \\
\hline
3   & Philosophiae naturalis principia mathematica & 1687 \\
\hline
4   & Opticks & 1704 \\
\hline
5   & Arithmetica Universalis  & 1707  \\
\hline
6   & An Historical Account of Two Notable Corruptions of Scripture  & 1754  \\
\hline   
\end{tabular} 
\newpage
\section*{Bibligrafia}
\begin{itemize}
\item Eric M. Rogers, Fizyka dla dociekliwych, tom 2, Astronomia, wydanie VI, Warszawa 1986, PWN, tłum. Marcin Kubiak, ISBN 83-01-02919-6, s. 3007.
\item Marian Grotowski, Newton; tom 1-3. Poznań: Księgarnia Św. Wojciecha, 1932-1933.
\item Frank E. Manuel, Portret Izaaka Newtona, seria: Na ścieżkach nauki, Prószyński i s-ka, Warszawa 1998.
\end{itemize}
\end{document}

